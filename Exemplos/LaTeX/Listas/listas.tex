\documentclass[12pt,a4paper]{article}

\usepackage[utf8]{inputenc}
\usepackage[brazil]{babel}
\usepackage{enumitem}

\title{Listas}
\author{Andrês}
\date{Para MA750 - Recursos Computacionais}

\begin{document}

\maketitle

\section{Lista Enumerada}

\begin{enumerate}
  \item Primeiro item
  \item Segundo item
  \item Terceiro item
  \item Quarto item
\end{enumerate}

\section{Lista Não Enumerada}

\begin{itemize}
  \item Primeiro item
  \item Segundo item
  \item Terceiro item
  \item Quarto item
\end{itemize}

\section{Lista Descritiva}

\begin{description}
  \item[Primeiro] Primeiro item
  \item[Segundo] Segundo item
  \item[Terceiro] Terceiro item
  \item[Quarto] Quarto item
  \item[] Item sem rótulo
\end{description}

\section{Listas Personalizadas}

\subsection{Lista Com Números Romanos}

\begin{enumerate}[label=\Roman*]
  \item Primeiro item
  \item Segundo item
  \item Terceiro item
  \item Quarto item
\end{enumerate}

\subsection{Lista Com Letras minúsculas}

\begin{enumerate}[label=\alph*]
  \item Primeiro item
  \item Segundo item
  \item Terceiro item
  \item Quarto item
\end{enumerate}

\subsection{Lista Com Letras maiúsculas}

\begin{enumerate}[label=\Alph*]
  \item Primeiro item
  \item Segundo item
  \item Terceiro item
  \item Quarto item
\end{enumerate}

\section{Lista dentro de Lista}

\begin{enumerate}
\item Essa é uma lista com listas
\item Aqui começa a lista aninhada
  \begin{enumerate}
  \item Primeiro item
  \item Segundo item
  \item Terceiro item
  \item Quarto item
  \end{enumerate}
\item Aqui termina a lista aninhada
\item Último item
\end{enumerate}

\section{Lista com Estilo Personalizado}

\begin{enumerate}[label=\textbf{Item \arabic*}]
  \item Primeiro item
  \item Segundo item
  \item Terceiro item
  \item Quarto item
\end{enumerate}


\end{document}

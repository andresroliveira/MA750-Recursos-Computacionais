\documentclass[a4paper, 12pt]{article}

%%%%%%%%%%%%%%%%%%%%%%%%%%%%%% Pacotes: básicos %%%%%%%%%%%%%%%%%%%%%%%%%%%%%%
\usepackage[utf8]{inputenc}
\usepackage[T1]{fontenc}
\usepackage{lmodern}
\usepackage[brazilian]{babel}
\usepackage{geometry}

%%%%%%%%%%%%%%%%%%%%%%%%%%%%%% Pacotes: links %%%%%%%%%%%%%%%%%%%%%%%%%%
\usepackage{hyperref}

%%%%%%%%%%%%%%%%%%%%%%%%%%%%%% Configurações: hyperref %%%%%%%%%%%%%%%%%%%%%%%%

\hypersetup{
    colorlinks=true,
    linkcolor=blue,
    filecolor=magenta,
}

%%%%%%%%%%%%%%%%%%%%%%%%%%%%%% Configurações: geometry %%%%%%%%%%%%%%%%%%%%%%%%

\geometry{
    left=20mm,
    right=20mm,
    top=20mm,
    bottom=20mm,
}

\title{Dicas}
\date{Andrês}
\author{Para MA750 - Recursos Computacionais}

\begin{document}
\maketitle

\tableofcontents

\section{Atalhos do Teclado}\label{atalhos-do-teclado}

\subsection{Windows}\label{windows}

\begin{itemize}
    \item \texttt{Alt + Tab}: Alternar janelas
    \item \texttt{Alt + F4}: Fechar Janela
    \item \texttt{Win + D}: Abrir área de trabalho
    \item \texttt{Win + E}: Abrir o Explorador de Arquivos.
    \item \texttt{Win + L}: Bloquear a tela.
    \item \texttt{Win + R}: Abrir a caixa de diálogo Executar.
    \item \texttt{Win + Seta Esquerda}: Mover a janela para a esquerda.
    \item \texttt{Win + Seta Direita}: Mover a janela para a direita.
    \item \texttt{Win + Seta Cima}: Maximizar a janela.
    \item \texttt{Win + Seta Baixo}: Minimizar a janela.
    \item \texttt{Win + Shift + Seta Esquerda}: Mover a janela para o monitor à esquerda.
    \item \texttt{Win + Shift + Seta Direita}: Mover a janela para o monitor à direita.
    \item \texttt{Win + Shift + Seta Cima}: Maximizar a janela verticalmente.
    \item \texttt{Win + Shift + Seta Baixo}: Restaurar a janela.
    \item \texttt{Win + Ctrl + D}: Criar uma nova área de trabalho.
    \item \texttt{Win + Ctrl + Seta Esquerda}: Mover para a área de trabalho à esquerda.
    \item \texttt{Win + Ctrl + Seta Direita}: Mover para a área de trabalho à direita.
    \item \texttt{Win + Ctrl + F4}: Fechar a área de trabalho atual.
    \item \texttt{Win + Tab}: Alternar entre as áreas de trabalho.
    \item \texttt{Win + Shift + Tab}: Alternar entre as áreas de trabalho na ordem inversa.
    \item \texttt{Win + 1, 2, 3, \ldots}: Abrir o programa na barra de tarefas.
\end{itemize}

\subsection{Navegador}\label{navegador}

\begin{itemize}
	\item \texttt{Ctrl + T}: Abrir uma nova aba.
	\item \texttt{Ctrl + W}: Fechar a aba atual.
	\item \texttt{Ctrl + F4}: Fechar a aba atual.
	\item \texttt{Ctrl + Shift + T}: Reabre a última aba fechada.
	\item \texttt{Ctrl + N}: Abrir uma nova janela.
	\item \texttt{Ctrl + Shift + N}: Abrir uma janela anônima.
	\item \texttt{Ctrl + Tab}: Alternar entre abas abertas.
	\item \texttt{Ctrl + Shift + Tab}: Alternar entre abas abertas na ordem inversa.
	\item \texttt{Ctrl + 1, 2, 3, \ldots}: Alternar entre abas pela ordem em que foram abertas.
	\item \texttt{Ctrl + L}: Seleciona o URL do site
	\item \texttt{Ctrl + F}: Abrir a barra de busca.
	\item \texttt{Ctrl + L}: Seleciona o URL do site
    \item \texttt{Ctrl + D}: Salvar a página atual nos favoritos.
    \item \texttt{Ctrl + H}: Abrir o histórico de navegação.
    \item \texttt{Ctrl + J}: Abrir o histórico de downloads.
    \item \texttt{Ctrl + Shift + Delete}: Limpar o histórico de navegação.
\end{itemize}

\subsection{Editores de Texto}\label{editores-de-texto}

\begin{itemize}
	\item \texttt{Ctrl + C}: Copiar
	\item \texttt{Ctrl + V}: Colar
	\item \texttt{Ctrl + X}: Recortar
	\item \texttt{Ctrl + Z}: Desfazer
	\item \texttt{Ctrl + Y}: Refazer
	\item \texttt{Ctrl + A}: Selecionar tudo
    \item \texttt{Ctrl + H}: Substituir
    \item \texttt{Ctrl + P}: Imprimir
    \item \texttt{Ctrl + S}: Salvar
    \item \texttt{Ctrl + O}: Abrir
	\item \texttt{Ctrl + Backspace}: Apagar a última palavra
	\item \texttt{Ctrl + Delete}: Apagar a próxima palavra
	\item \texttt{Home}: Começo da Linha
	\item \texttt{End}: Fim da Linha
	\item \texttt{Ctrl + Home}: Começo do documento
	\item \texttt{Ctrl + End}: Fim do documento
	\item \texttt{Ctrl + F}: Abrir a barra de busca.
	\item \texttt{Ctrl + Seta Esquerda}: volta uma palavra
	\item \texttt{Ctrl + Seta Direita}: avança uma palavra
	\item \texttt{Ctrl + Shift + Seta Esquerda}: seleciona palavra anterior
	\item \texttt{Ctrl + Shift + Seta Direita}: seleciona próxima palavra
	\item \texttt{Ctrl + Shift + Home}: Seleciona do cursor até o começo da linha
	\item \texttt{Ctrl + Shift + End}: Seleciona do cursor até o fim da linha
	\item \texttt{Ctrl + Shift + Seta Cima}: Seleciona a linha acima
	\item \texttt{Ctrl + Shift + Seta Baixo}: Seleciona a linha abaixo
	\item \texttt{Ctrl + Shift + Home}: Seleciona do cursor até o começo do documento
	\item \texttt{Ctrl + Shift + End}: Seleciona do cursor até o fim do documento
\end{itemize}

\subsection{Overleaf}\label{overleaf}

\begin{itemize}
	\item \texttt{Ctrl + Alt + Seta Abaixo}: Insere um cursor abaixo
	\item \texttt{Ctrl + Alt + Seta Acima}: Insere um cursor acima
	\item \texttt{Ctrl + D}: Deleta a linha atual
	\item \texttt{Ctrl + Shift + L}: Vai para uma linha específica
	\item \texttt{Ctrl + ; }: Comenta/descomenta a linha
	\item \texttt{Ctrl + / }: Comenta/descomenta a linha
	\item \texttt{Ctrl + U}: Muda para maiusculas
	\item \texttt{Ctrl + Shift + U}: Muda para minusculas
	\item \texttt{Ctrl + Espaço}: Autocompleta
	\item \texttt{Ctrl + B}: Negrito
	\item \texttt{Ctrl + I}: Itálico
	\item \texttt{Alt + Seta Abaixo}: Move a linha para baixo
	\item \texttt{Alt + Seta Acima}: Move a linha para cima
	\item \texttt{Ctrl + Enter}: Recompila o documento
	\item \texttt{Ctrl + S}: Recompila o documento
	\item \texttt{Ctrl + .}: Recompila o documento
\end{itemize}

\subsection{Overleaf (avançado)}

\begin{itemize}
    \item Habilitar \textit{VIM Mode}
    \item Habilitar \textit{EMACS Mode}
\end{itemize}

\section{\LaTeX}\label{latex}

\subsection{Distribuições}\label{distribuicoes}

\begin{itemize}
    \item MiKTeX:

        MiKTeX é uma distribuição de \LaTeX para Windows. Com um gerenciador de pacotes, o MiKTeX facilita a instalação de pacotes e fontes. Seu site oficial é \url{https://miktex.org/}. Usa um sistema de instalação por demanda, ou seja, instala os pacotes conforme são necessários. Seu interface gráfica é o MiKTeX Console.


    \item TeX Live:

        TeX Live é uma distribuição de \LaTeX para Linux, Windows e MAC. Com um gerenciador de pacotes, o TeX Live facilita a instalação de pacotes e fontes. Seu site oficial é \url{https://www.tug.org/texlive/}. Usa um sistema de instalação variado, ou seja, pode instalar todos os pacotes disponíveis, somente os básicos. Seu método é via \texttt{tlmgr}, o gerenciador de pacotes do TeX Live.

\end{itemize}

\subsection{Editores}\label{editores}

\begin{itemize}
    \item TeXworks: Editor de \LaTeX  padrão do MiKTeX.
    \item TeXstudio: Editor de \LaTeX  multiplataforma.

    \item Overleaf: Editor de \LaTeX  online.
\end{itemize}


\subsection{Compiladores}\label{compiladores}


\begin{itemize}

    \item \texttt{pdflatex}: Compilador de \LaTeX  para PDF.
    \item \texttt{xelatex}: Compilador de \LaTeX  para PDF com suporte a fontes do sistema.
    \item \texttt{lualatex}: Compilador de \LaTeX  para PDF com suporte a Lua.
\end{itemize}


\subsection{Pacotes}\label{pacotes}


\begin{itemize}

    \item \texttt{babel}: Pacote de idiomas.
    \item \texttt{inputenc}: Pacote de codificação de caracteres.
    \item \texttt{fontenc}: Pacote de codificação de fontes.
    \item \texttt{geometry}: Pacote de configuração de margens.
    \item \texttt{hyperref}: Pacote de links.

    \item \texttt{graphicx}: Pacote de imagens.
    \item \texttt{amsmath}: Pacote de matemática.
    \item \texttt{amssymb}: Pacote de símbolos matemáticos.

    \item \texttt{amsfonts}: Pacote de fontes matemáticas.
    \item \texttt{listings}: Pacote de códigos.
    \item \texttt{tikz}: Pacote de desenhos.
    \item \texttt{pgfplots}: Pacote de gráficos.
    \item \texttt{multirow}: Pacote de tabelas.
    \item \texttt{multicol}: Pacote de colunas.
    \item \texttt{enumitem}: Pacote de listas.
    \item \texttt{fancyhdr}: Pacote de cabeçalhos e rodapés.
\end{itemize}

\subsection{Comandos}\label{comandos}

\begin{itemize}
    \item \texttt{\textbackslash documentclass}: Define a classe do documento.
    \item \texttt{\textbackslash usepackage}: Importa pacotes.
    \item \texttt{\textbackslash title}: Define o título do documento.
    \item \texttt{\textbackslash author}: Define o autor do documento.
    \item \texttt{\textbackslash date}: Define a data do documento.
    \item \texttt{\textbackslash begin\{document\}}: Inicia o documento.
    \item \texttt{\textbackslash end\{document\}}: Finaliza o documento.
    \item \texttt{\textbackslash maketitle}: Cria o título do documento.
    \item \texttt{\textbackslash section}: Cria uma seção.
    \item \texttt{\textbackslash subsection}: Cria uma subseção.
    \item \texttt{\textbackslash subsubsection}: Cria uma subsubseção.
    \item \texttt{\textbackslash textbf}: Texto em negrito.
    \item \texttt{\textbackslash textit}: Texto em itálico.
    \item \texttt{\textbackslash texttt}: Texto em monoespaçado.
    \item \texttt{\textbackslash textcolor}: Texto colorido.
    \item \texttt{\textbackslash href}: Link.
    \item \texttt{\textbackslash url}: Link.
\end{itemize}


\section{Comandos Básicos do Terminal}\label{comandos-basicos-do-terminal}

\subsection{CMD-Windows}\label{cmd-windows}

\begin{itemize}
    \item \texttt{dir}: Lista os arquivos e diretórios do diretório atual.
    \item \texttt{cd}: Muda de diretório.
    \item \texttt{mkdir}: Cria um diretório.
    \item \texttt{echo}: Mostra o conteúdo de um arquivo.
    \item \texttt{copy}: Copia um arquivo.
    \item \texttt{move}: Move um arquivo.
    \item \texttt{del}: Remove um arquivo.
    \item \texttt{rd}: Remove um diretório.
    \item \texttt{cls}: Limpa a tela.
    \item \texttt{exit}: Fecha o terminal.
\end{itemize}

\subsection{Bash-Linux}\label{bash-linux}

\begin{itemize}

    \item \texttt{ls}: Lista os arquivos e diretórios do diretório atual.
    \item \texttt{pwd}: Mostra o diretório atual.
    \item \texttt{cd}: Muda de diretório.
    \item \texttt{mkdir}: Cria um diretório.
    \item \texttt{touch}: Cria um arquivo.
    \item \texttt{rm}: Remove um arquivo.
    \item \texttt{rm -r}: Remove um diretório.
    \item \texttt{cp}: Copia um arquivo.
    \item \texttt{mv}: Move um arquivo.
    \item \texttt{cat}: Mostra o conteúdo de um arquivo.
    \item \texttt{less}: Mostra o conteúdo de um arquivo de forma paginada.

    \item \texttt{head}: Mostra as primeiras linhas de um arquivo.
    \item \texttt{tail}: Mostra as últimas linhas de um arquivo.

    \item \texttt{grep}: Procura por um padrão em um arquivo.
    \item \texttt{find}: Procura por arquivos e diretórios.
    \item \texttt{man}: Mostra o manual de um comando.
    \item \texttt{clear}: Limpa a tela.
    \item \texttt{exit}: Fecha o terminal.
\end{itemize}

\subsection{GIT}\label{git}

\begin{itemize}
    \item \texttt{git init}: Inicia um repositório.
    \item \texttt{git clone}: Clona um repositório.
    \item \texttt{git add}: Adiciona arquivos ao stage.
    \item \texttt{git commit}: Salva as alterações.
    \item \texttt{git push}: Envia as alterações para o repositório remoto.
    \item \texttt{git pull}: Atualiza o repositório local.
    \item \texttt{git status}: Mostra o status do repositório.
    \item \texttt{git log}: Mostra o histórico de commits.
    \item \texttt{git branch}: Mostra as branches.
    \item \texttt{git checkout}: Muda de branch.
    \item \texttt{git merge}: Une duas branches.
    \item \texttt{git rebase}: Reescreve o histórico.
    \item \texttt{git reset}: Volta para um commit anterior.
    \item \texttt{git revert}: Desfaz um commit.
\end{itemize}

\subsection{Conda}\label{conda}

\begin{itemize}
    \item \texttt{conda create}: Cria um ambiente virtual.
    \item \texttt{conda activate}: Ativa um ambiente virtual.
    \item \texttt{conda deactivate}: Desativa um ambiente virtual.
    \item \texttt{conda list}: Lista os pacotes instalados.
    \item \texttt{conda install}: Instala um pacote.
    \item \texttt{conda remove}: Remove um pacote.
    \item \texttt{conda update}: Atualiza um pacote.
    \item \texttt{conda search}: Procura por um pacote.
    \item \texttt{conda env export}: Exporta um ambiente virtual.
    \item \texttt{conda env create}: Cria um ambiente virtual a partir de um arquivo.
\end{itemize}

\subsection{Pip}\label{pip}

\begin{itemize}
    \item \texttt{pip install}: Instala um pacote.
    \item \texttt{pip uninstall}: Desinstala um pacote.
    \item \texttt{pip freeze}: Lista os pacotes instalados.
    \item \texttt{pip search}: Procura por um pacote.
    \item \texttt{pip show}: Mostra informações sobre um pacote.
\end{itemize}

\section{Wolfram Mathematica}\label{wolfram-mathematica}

\subsection{Funções}\label{funcoes}

\subsubsection{Gráficos}\label{graficos}

\begin{itemize}
    \item \texttt{Plot}: Gráfico de uma função.
    \item \texttt{Plot3D}: Gráfico de uma função tridimensional.
    \item \texttt{ContourPlot}: Gráfico de contorno de uma função.
    \item \texttt{ContourPlot3D}: Gráfico de contorno tridimensional de uma função.
    \item \texttt{DensityPlot}: Gráfico de densidade de uma função.
    \item \texttt{DensityPlot3D}: Gráfico de densidade tridimensional de uma função.
    \item \texttt{VectorPlot}: Gráfico de um campo vetorial.
    \item \texttt{ParametricPlot}: Gráfico de uma curva paramétrica.
    \item \texttt{ParametricPlot3D}: Gráfico de uma curva paramétrica tridimensional.
    \item \texttt{RegionPlot}: Gráfico de uma região.
    \item \texttt{RegionPlot3D}: Gráfico de uma região tridimensional.
    \item \texttt{Graphics}: Gráfico personalizado.
    \item \texttt{Show}: Mostra um gráfico.
    \item \texttt{Manipulate}: Cria uma interface interativa.
    \item \texttt{Export}: Exporta um gráfico.
    \item \texttt{ArrayPlot}: Gráfico de uma matriz.
    \item \texttt{BarChart}: Gráfico de barras.
    \item \texttt{Histogram}: Histograma.
    \item \texttt{PieChart}: Gráfico de pizza.
    \item \texttt{ListPlot}: Gráfico de pontos.
    \item \texttt{ProbabilityPlot}: Gráfico de probabilidade.
\end{itemize}

\subsubsection{Cálculos}\label{calculos}

\begin{itemize}
    \item \texttt{Solve}: Resolve uma equação.
    \item \texttt{DSolve}: Resolve uma equação diferencial.
    \item \texttt{Integrate}: Calcula uma integral.
    \item \texttt{Sum}: Calcula uma soma.
    \item \texttt{Limit}: Calcula um limite.
    \item \texttt{Simplify}: Simplifica uma expressão.
    \item \texttt{Expand}: Expande uma expressão.
    \item \texttt{Factor}: Fatora uma expressão.
\end{itemize}

\subsubsection{Matrizes}\label{matrizes}

\begin{itemize}
    \item \texttt{MatrixForm}: Mostra uma matriz.
    \item \texttt{Eigenvalues}: Calcula os autovalores de uma matriz.
    \item \texttt{Eigenvectors}: Calcula os autovetores de uma matriz.
    \item \texttt{Det}: Calcula o determinante de uma matriz.
    \item \texttt{Inverse}: Calcula a inversa de uma matriz.
    \item \texttt{Transpose}: Calcula a transposta de uma matriz.
    \item \texttt{Cross}: Calcula o produto vetorial.
    \item \texttt{Dot}: Calcula o produto escalar.
    \item \texttt{Norm}: Calcula a norma de um vetor.
    \item \texttt{VectorAngle}: Calcula o ângulo entre dois vetores.
    \item \texttt{MatrixPower}: Calcula a potência de uma matriz.
    \item \texttt{DiagonalMatrix}: Cria uma matriz diagonal.
    \item \texttt{LinearSolve}: Resolve um sistema linear.
    \item \texttt{SingularValueDecomposition}: Calcula a decomposição em valores singulares.
    \item \texttt{CholeskyDecomposition}: Calcula a decomposição de Cholesky.
    \item \texttt{QRDecomposition}: Calcula a decomposição QR.
    \item \texttt{SingularValueList}: Lista os valores singulares.
    \item \texttt{SingularValuePlot}: Gráfico dos valores singulares.
\end{itemize}

\subsubsection{Probabilidade e Estatística}\label{probabilidade-e-estatistica}

\begin{itemize}
    \item \texttt{RandomReal}: Gera números aleatórios reais.
    \item \texttt{RandomInteger}: Gera números aleatórios inteiros.
    \item \texttt{RandomVariate}: Gera números aleatórios de uma distribuição.
    \item \texttt{RandomChoice}: Escolhe um elemento aleatório de uma lista.
    \item \texttt{RandomSample}: Escolhe uma amostra aleatória de uma lista.
    \item \texttt{RandomPermutation}: Gera uma permutação aleatória de uma lista.
    \item \texttt{RandomSeed}: Define a semente aleatória.
    \item \texttt{Mean}: Calcula a média.

    \item \texttt{Median}: Calcula a mediana.
    \item \texttt{Mode}: Calcula a moda.
    \item \texttt{Variance}: Calcula a variância.
    \item \texttt{StandardDeviation}: Calcula o desvio padrão.
    \item \texttt{BinCounts}: Conta os elementos em intervalos.
    \item \texttt{HistogramList}: Lista os intervalos de um histograma.
    \item \texttt{NormalDistribution}: Distribuição normal.
    \item \texttt{PoissonDistribution}: Distribuição de Poisson.
    \item \texttt{ExponentialDistribution}: Distribuição exponencial.
    \item \texttt{ChiSquareDistribution}: Distribuição qui-quadrado.
    \item \texttt{StudentTDistribution}: Distribuição t de Student.
    \item \texttt{UniformDistribution}: Distribuição uniforme.
    \item \texttt{MultinormalDistribution}: Distribuição multinormal.
    \item \texttt{MultinomialDistribution}: Distribuição multinomial.
\end{itemize}

\subsubsection{Processamento de Imagens}\label{processamento-de-imagens}

\begin{itemize}
    \item \texttt{Image}: Importa uma imagem.
    \item \texttt{ImageAdjust}: Ajusta o brilho e contraste de uma imagem.
    \item \texttt{ImageCrop}: Corta uma imagem.
    \item \texttt{ImageResize}: Redimensiona uma imagem.
    \item \texttt{EdgeDetect}: Detecta bordas em uma imagem.
    \item \texttt{ImageConvolve}: Convolução de uma imagem.
\end{itemize}


\subsubsection{Processamento de Áudio}\label{processamento-de-audio}

\begin{itemize}
    \item \texttt{Audio}: Importa um arquivo de áudio.
    \item \texttt{AudioCapture}: Captura um áudio.
    \item \texttt{AudioPlay}: Reproduz um áudio.
    \item \texttt{AudioRecord}: Grava um áudio.
    \item \texttt{AudioPlot}: Plota um áudio.
    \item \texttt{Spectrogram}: Espectrograma de um áudio.
    \item \texttt{AudioConvolve}: Convolução de um áudio.
\end{itemize}

\subsubsection{Soluções Numéricas}\label{solucoes-numericas}

\begin{itemize}
    \item \texttt{NDSolve}: Resolve uma equação diferencial numéricamente.
    \item \texttt{FindRoot}: Encontra a raiz de uma equação.
    \item \texttt{FindMinimum}: Encontra o mínimo de uma função.
    \item \texttt{FindMaximum}: Encontra o máximo de uma função.
    \item \texttt{FindFit}: Encontra o melhor ajuste de uma função.
    \item \texttt{NMinimize}: Minimiza uma função.
    \item \texttt{NMaximize}: Maximiza uma função.
    \item \texttt{NIntegrate}: Calcula uma integral numéricamente.
    \item \texttt{NSum}: Calcula uma soma numéricamente.
    \item \texttt{Interpolation}: Interpola uma função.
\end{itemize}

\subsubsection{Teoria dos Números e Combinatória}\label{teoria-dos-numeros-e-combinatoria}

\begin{itemize}
    \item \texttt{FactorInteger}: Fatora um número em primos.
    \item \texttt{PrimeQ}: Verifica se um número é primo.
    \item \texttt{PrimePi}: Conta os números primos menores que um número.
    \item \texttt{Prime}: Encontra o enésimo número primo.
    \item \texttt{NextPrime}: Encontra o próximo número primo.
    \item \texttt{Divisors}: Lista os divisores de um número.
    \item \texttt{GCD}: Calcula o máximo divisor comum.
    \item \texttt{LCM}: Calcula o mínimo múltiplo comum.
    \item \texttt{Permutations}: Permutações de uma lista.
    \item \texttt{Combinations}: Combinações de uma lista.
    \item \texttt{Binomial}: Coeficiente binomial.
    \item \texttt{Fibonacci}: Sequência de Fibonacci.
    \item \texttt{Permutations}: Permutações de uma lista.
\end{itemize}

\subsubsection{Geometria}\label{geometria}

\begin{itemize}
    \item \texttt{Point}: Ponto.
    \item \texttt{Line}: Linha.
    \item \texttt{Circle}: Círculo.
    \item \texttt{Polygon}: Polígono.
    \item \texttt{Triangle}: Triângulo.
    \item \texttt{Rectangle}: Retângulo.
    \item \texttt{Disk}: Disco.
    \item \texttt{Cuboid}: Cuboide.
    \item \texttt{Cylinder}: Cilindro.
    \item \texttt{Cone}: Cone.
    \item \texttt{Sphere}: Esfera.
\end{itemize}


\subsection{Documentação}\label{documentacao}

\begin{itemize}
    \item \texttt{?função}: Mostra a documentação de uma função.
    \item \texttt{??função}: Mostra a documentação detalhada de uma função.
    \item \texttt{Wolfram Language Documentation}: Documentação online.  Acesse o site em \url{https://reference.wolfram.com/language/}.
\end{itemize}



\end{document}
